
{\color{red} \bf Warning:} This document is written to explain the major
functions of
\pkg{pmclust}~\citep{Chen2012pmclustpackage}, version 0.1-4.
Every effort will be made to insure future versions are consistent with
these instructions, but new features in later versions may not be explained
in this document.


\section[Introduction]{Introduction}
\label{sec:introduction}
\addcontentsline{toc}{section}{\thesection. Introduction}

This is a quick guide to the package \pkg{pmclust} for
parallel model-based clustering.
We will cover how to perform parallel EM algorithm in 
a single program multiple data (SPMD) programming model,
and a master and workers programming model.
The main function \code{em.step} will cluster
data in to $K$ different groups based on a finite mixture Gaussian
model with unstructured dispersions.

Information about the detail functionality of this package,
and any changes in future versions can be found on our website
``Programming with Big Data in R'' at
\url{http://r-pbd.org/}.
More discussions about parallel computing in \proglang{R}~\citep{Rcore}
can be found in \citet{Schmidberger2009}.
The system requirement for \pkg{pmclust} is described next.
In Section~\ref{sec:example}, examples are demonstrated.
In Section~\ref{sec:algorithm}, algorithms implemented in \pkg{pmclust}
are introduced. In Section~\ref{sec:discussion}, a few performance issues
are discussed.




\subsection[System Requirement]{System Requirement}
\label{sec:system_requirement}
\addcontentsline{toc}{subsection}{\thesubsection. System Requirement}

The \pkg{pmclust} is mainly developed and tested under Linux operating systems
(\url{http://en.wikipedia.org/wiki/Linux}).
The major computing environment for \pkg{pmclust} requires MPI systems
(\url{http://en.wikipedia.org/wiki/Message_Passing_Interface}), such as
LAM/MPI (\url{http://www.lam-mpi.org/}) or
OpenMPI (\url{http://www.open-mpi.org/}).
Both of SPMD and master and workers programming models are also tested.

Other operating systems such as Mac or Windows are also possible to run the
\pkg{pmclust} if MPI systems and \pkg{pbdMPI}~\citep{Chen2012pbdMPIpackage}
are correctly installed, and the instruction can be found in
the vignette of \pkg{pbdMPI}~\citep{Chen2012pbdMPIvignette}.
For master and workers models,
it is also possible to run the \pkg{pmclust} within \pkg{Rmpi}~\citep{Yu2010}
via \pkg{pbdMPI}.

Note that \pkg{Rmpi}~\citep{Yu2010}
requires more complicated settings for running on
some MPI systems under the mater and workers programming model.
See \pkg{Rmpi}'s website for details at
\url{http://www.stats.uwo.ca/faculty/yu/Rmpi/}.


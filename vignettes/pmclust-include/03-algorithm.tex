\section[Algorithm]{Algorithm}
\label{sec:algorithm}
\addcontentsline{toc}{section}{\thesection. Algorithm}

Five algorithms are implemented in this packages including:
EM~\citep{Dempster1977}, AECM~\citep{Meng1997}, APECM~\citep{Chen2011},
APECMa~\citep{Chen2013}, and K-means~\citep{Lloyd1982}.
The EM-like algorithms (EM, AECM, APECM, and APECMa) are for model-based
clusterings~\citep{Fraley2002},
while K-means is a distance based clustering algorithm.

In general, AECM, APECM, and APECMa will have quick convergent rate in terms
of few iterations than EM. Since the more E-steps is updated,
the more log likelihood is increased.
But, AECM and APECMa may take long computing time
than EM if the M-step has analytic solutions. In this situation, the majority
of computing time in one iteration mostly spend on the E-step. So, the more
E-steps called, the more time spent for an entire iteration.

While the analytic solutions are not available for the M-step, the
optimization routines are required for maximization of complete log
likelihoods. Note that the solutions only exists in some simplified cases,
and is not solvable in general.
In this situation, the M-step may slow down hugely
EM iterations and cost more computing time than the E-step.
Considering faster convergent rate, like AECM or APECM,
is a better choice~\citep{Chen2011}.

The APECMa takes benefits in both of computing time and convergent rate.
In the first situation, APECMa has the same order of E-step computations
as the EM has, so it can converge faster than EM and has less computing time.
In the second situation, APECMa has a similar convergent rate as AECM
and APECMa, so it can converge as efficient as the both algorithms and
has less computing time among all other EM-like algorithms.

The K-means probably is the fast among all algorithms since it is restricted
in a very simple model. However, the initialization procedure~\citep{Maitra2009}
based on this algorithm may capture the skeleton of data, and may boost
the EM-like algorithms and improve convergent results.

